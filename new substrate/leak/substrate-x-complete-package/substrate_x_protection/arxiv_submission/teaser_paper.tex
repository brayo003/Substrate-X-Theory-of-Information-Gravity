\documentclass[12pt]{article}
\usepackage{amsmath}
\usepackage{amssymb}

\title{Substrate X Theory: Reproduction of Key Gravitational Tests}
\author{Mwirigi Bryan Kenda \\ 
Independent Researcher \\
Nakuru County, Kenya \\
\texttt{briankenda99@gmail.com}}
\date{\today}

\begin{document}

\maketitle

\begin{abstract}
This brief communication presents initial results from Substrate X Theory, a novel framework for gravitational phenomena. The theory successfully reproduces three classical tests of General Relativity: Mercury's perihelion precession (42.9''/century), gravitational lensing (1.75 arcseconds), and binary pulsar orbital decay ($-2.40 \times 10^{-12}$ s/s), matching observational data. A complete theoretical framework with physical interpretation will be presented in subsequent publications.
\end{abstract}

\section{Introduction}
The empirical success of General Relativity (GR) is well-established, though questions remain about the physical mechanism underlying gravitational interactions. This communication reports that Substrate X Theory, a new approach based on information dynamics in a physical substrate, reproduces key gravitational tests with precision matching GR predictions.

\section{Key Results}
\subsection{Mercury's Perihelion Precession}
The theory predicts a perihelion advance of $42.9''$ per century, matching the observed value of $43.0''$ per century within measurement uncertainty.

\subsection{Gravitational Lensing} 
For light bending by the Sun, the theory predicts $1.75$ arcseconds, consistent with both GR and experimental measurements.

\subsection{Binary Pulsar Orbital Decay}
The orbital period derivative is predicted as $-2.40 \times 10^{-12}$ s/s, matching the Hulse-Taylor binary pulsar observation of $-2.405 \times 10^{-12}$ s/s.

\begin{table}[h]
\centering
\begin{tabular}{lccc}
\hline
Test & Prediction & Observed & Status \\
\hline
Mercury Precession & $42.9''$/century & $43.0''$/century & Match \\
Gravitational Lensing & 1.75 arcsec & $1.75 \pm 0.05$ arcsec & Match \\
Binary Pulsar Decay & $-2.40 \times 10^{-12}$ s/s & $-2.405 \times 10^{-12}$ s/s & Match \\
\hline
\end{tabular}
\caption{Comparison of theory predictions with experimental results}
\end{table}

\section{Theoretical Framework}
The gravitational interaction in Substrate X Theory takes the form:
\begin{equation}
\mathbf{F}_{\text{grav}} = k s \mathbf{v}_{\text{sub}}
\end{equation}
where $s$ represents a fundamental field and $\mathbf{v}_{\text{sub}}$ characterizes substrate dynamics. The complete theoretical framework, including derivation and physical interpretation, will be presented in an extended publication.

\section{Conclusion}
Substrate X Theory reproduces three key gravitational tests with precision matching General Relativity. The full theoretical framework, including novel predictions and physical interpretation, is in preparation for subsequent publication.

\end{document}
