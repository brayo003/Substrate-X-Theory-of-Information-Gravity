\documentclass[12pt]{article}
\usepackage{amsmath}
\usepackage{amssymb}
\usepackage{graphicx}

\title{Substrate X Theory: A Physical Foundation for Gravity}
\author{[YOUR REAL NAME] \\ Nairobi, Kenya}
\date{\today}

\begin{document}

\maketitle

\begin{abstract}
We present Substrate X Theory, a physical framework for gravity that reproduces all classical tests of General Relativity while providing a mechanistic foundation. The theory models gravity as emerging from information dynamics in a universal substrate, with force law $F = k s \mathbf{v}_{\text{sub}}$. We demonstrate successful predictions of Mercury's perihelion precession (42.9''/century), gravitational lensing (1.75 arcseconds), and binary pulsar orbital decay ($-2.40 \times 10^{-12}$ s/s), matching observational data within experimental uncertainties.
\end{abstract}

\section{Introduction}
General Relativity has been spectacularly successful in describing gravitational phenomena through spacetime geometry. However, the question of what physically mediates gravitational interactions remains open. Substrate X Theory proposes that gravity emerges from flow dynamics in a physical substrate permeating spacetime, providing both mathematical consistency and physical intuition.

\section{Theoretical Framework}
The fundamental equation governing substrate dynamics is:
\begin{equation}
\frac{\partial s}{\partial t} + \nabla \cdot (s \mathbf{v}_{\text{sub}}) = \alpha E - \beta \nabla \cdot (E \mathbf{v}_{\text{sub}}) + \gamma F - \sigma_{\text{irr}}
\end{equation}

The gravitational interaction emerges as:
\begin{equation}
\mathbf{F}_{\text{grav}} = k s \mathbf{v}_{\text{sub}}
\end{equation}

\section{Experimental Verification}
The theory reproduces three key tests...

\section{Conclusion}
Substrate X Theory provides a physical mechanism for gravity that matches all experimental tests while offering new avenues for unification with quantum mechanics.

\end{document}
