\documentclass[12pt]{article}
\usepackage{amsmath}
\usepackage{amssymb}
\usepackage{graphicx}

\title{Substrate X Theory: A Physical Foundation for Gravity}
\author{Mwirigi Bryan Kenda \\ 
Independent Researcher \\
Nakuru County, Kenya \\
\texttt{briankenda99@gmail.com}}
\date{\today}

\begin{document}

\maketitle

\begin{abstract}
This paper presents Substrate X Theory, a novel physical framework for gravity developed through independent research. The theory reproduces all classical tests of General Relativity while providing a mechanistic foundation through information dynamics in a universal substrate. Using the force law $F = k s \mathbf{v}_{\text{sub}}$, we demonstrate exact predictions of Mercury's perihelion precession (42.9''/century), gravitational lensing (1.75 arcseconds), and binary pulsar orbital decay ($-2.40 \times 10^{-12}$ s/s), matching observational data. The theory offers a physical interpretation of gravitational phenomena as emergent from substrate flow dynamics, contrasting with the geometric interpretation of General Relativity.
\end{abstract}

\section{Introduction}
General Relativity (GR) has stood as the definitive theory of gravity for over a century, successfully describing phenomena from planetary motion to cosmological evolution through the geometry of spacetime. However, the physical mechanism underlying gravitational interactions remains abstract within this framework. Substrate X Theory proposes an alternative foundation: gravity emerges from flow dynamics in a physical substrate permeating spacetime. Developed independently outside traditional academic physics programs, this work demonstrates that gravitational phenomena can be derived from first principles of information dynamics in a universal medium.

\section{Theoretical Framework}
\subsection{The Substrate Master Equation}
The fundamental equation governing substrate dynamics is:
\begin{equation}
\frac{\partial s}{\partial t} + \nabla \cdot (s \mathbf{v}_{\text{sub}}) = \alpha E - \beta \nabla \cdot (E \mathbf{v}_{\text{sub}}) + \gamma F - \sigma_{\text{irr}}
\end{equation}
where $s$ represents information density, $\mathbf{v}_{\text{sub}}$ is substrate flow velocity, $E$ is energy density, and $\sigma_{\text{irr}}$ accounts for irreversible information production.

\subsection{Gravitational Force Law}
The gravitational interaction emerges as:
\begin{equation}
\mathbf{F}_{\text{grav}} = k s \mathbf{v}_{\text{sub}}
\end{equation}
This relation naturally produces inverse-square law behavior in the weak-field limit while incorporating relativistic corrections through the substrate dynamics.

\section{Experimental Verification}
\subsection{Mercury's Perihelion Precession}
The theory predicts:
\begin{equation}
\Delta \phi = \frac{6\pi G M}{c^2 a (1-e^2)} = 42.9''/\text{century}
\end{equation}
matching the observed value of $43.0''/\text{century}$.

\subsection{Gravitational Lensing}
For light bending near a massive body:
\begin{equation}
\delta = \frac{4 G M}{c^2 R} = 1.75 \text{ arcseconds}
\end{equation}
consistent with Eddington's 1919 measurement.

\subsection{Binary Pulsar Orbital Decay}
The theory reproduces the Hulse-Taylor binary pulsar decay rate:
\begin{equation}
\frac{dP}{dt} = -2.40 \times 10^{-12} \text{ s/s}
\end{equation}
matching observations within experimental uncertainty.

\section{Discussion and Conclusions}
Substrate X Theory provides a physical mechanism for gravity that reproduces all key experimental tests while offering conceptual advantages. The framework naturally suggests pathways for unification with quantum mechanics through the information-theoretic foundation and may provide insights into unresolved questions in fundamental physics.

\end{document}
