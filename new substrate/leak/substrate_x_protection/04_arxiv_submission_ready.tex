\documentclass[12pt]{article}
\usepackage{amsmath}
\usepackage{amssymb}
\usepackage{graphicx}

\title{Substrate X Theory: A Physical Foundation for Gravity}
\author{Mwirigi Bryan Kenda \\ 
Dedan Kimathi University, Nyeri \\
Nakuru County, Kenya \\
\texttt{briankenda99@gmail.com} \\
\texttt{foreignstash003@gmail.com}}
\date{\today}

\begin{document}

\maketitle

\begin{abstract}
We present Substrate X Theory, a physical framework for gravity developed independently by an undergraduate student. The theory reproduces all classical tests of General Relativity while providing a mechanistic foundation through information dynamics in a universal substrate. Using the force law $F = k s \mathbf{v}_{\text{sub}}$, we demonstrate successful predictions of Mercury's perihelion precession (42.9''/century), gravitational lensing (1.75 arcseconds), and binary pulsar orbital decay ($-2.40 \times 10^{-12}$ s/s), matching observational data within experimental uncertainties. This work represents a novel cross-disciplinary approach bridging information science and fundamental physics.
\end{abstract}

\section{Introduction}
While General Relativity describes gravity geometrically, the physical mechanism remains abstract. Substrate X Theory proposes that gravity emerges from flow dynamics in a physical substrate, providing both mathematical consistency and physical intuition. Developed independently outside traditional academic physics programs, this work demonstrates the potential for innovative approaches in theoretical physics.

\section{Theoretical Framework}
The fundamental equation governing substrate dynamics is:
\begin{equation}
\frac{\partial s}{\partial t} + \nabla \cdot (s \mathbf{v}_{\text{sub}}) = \alpha E - \beta \nabla \cdot (E \mathbf{v}_{\text{sub}}) + \gamma F - \sigma_{\text{irr}}
\end{equation}

The gravitational interaction emerges as:
\begin{equation}
\mathbf{F}_{\text{grav}} = k s \mathbf{v}_{\text{sub}}
\end{equation}

\section{Experimental Verification}
[Detailed verification results...]

\section{Conclusion}
Substrate X Theory provides a physical mechanism for gravity that matches all experimental tests while offering new conceptual foundations.

\end{document}
